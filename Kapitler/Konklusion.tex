\thispagestyle{fancy}
\chapter{Konklusion}
\label{chp:Konklusion}
Med udgangspunkt i kravspecifikationen, vil der blive konkluderet på hvorvidt udviklingsprocesen og delimplementeringerne samlet opfylder kravene. Afslutningsvist indeholder kapitlet indegolde en reflektion over læringsprocessen og den tilegnede viden, ift. læringsmålene for faget og de formelle krav.

I kravspecifikationen er de overordnede krav blevet prioriteret i en MoSCoW-analyse. På baggrund af denne analyse blev der udarbejdet user stories for "Must" og "Should" funktionalitet. Disse user stories udgjorde kravene til app'en. 

Det er lykkedes i dette projekt at udarbejde en fuldt funktionel Android applikation som stort set opfylder samtlige af de prioriterede krav. Det er muligt at logge ind i appen ved hjælp af Facebook. Man kan starte en gåtur og følge med i ruten på et kort løbende. Samtidig kan brugeren se hvor mange km der er gået og hvor man penge det svarer til i donationer. Når brugeren er færdig med at gå, kan turen afsluttes i appen og brugen har mulighed for at se og dele resultatet med andre brugere på Facebook. Det er lykkedes at sende en HTTP anmodning til vores Restfull service ved Google Cloud. Det er ikke lykkedes at konfigurer en SQL database på google cloud således at vores anmodninger til servicen gemmes.

Der har i udviklingsprocessen været et stringent fokus på kravspecifikationen, hvilket betyder at funktionalitet og teknologi har været prioriteret højere end design og indhold. Det betyder også at appen på nuværrende tildpunkt ikke er produktionsklar, hvilket iøvrigt heller ikke har været hverken et krav eller ambition.




%Konkludere på 
%- MoScow
%- User Stories (Krav)
%- Læringsmål for faget
%- Krav til projektet
%- Reflektion over egen læring og ideer til hvordan processen kunne have været håndteret anderledes.