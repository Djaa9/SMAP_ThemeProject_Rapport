\thispagestyle{fancy}
\chapter{Kravspecifikation}
\label{chp:kravspec}

I følgende kapitel redegøres for kravene som opstilles for projektet. Kravene er baseret på en prioritering, som er vurderes på baggrund af en MoSCoW analyse, som ses i tabel \ref{tab:MosCoW}. Der vil i udviklingsforløbet først fokuseres på MUST kolonnen, herefter SHOULD osv. 


\begin{table}[h]
\begin{tabular}{|>{\rr}p{3.2cm}|>{\rr}p{3.2cm}|>{\rr}p{3.2cm}|p{3.2cm}|} 
\hline
\rowcolor{ThemeColor!80}  
\rowcolor{ThemeColor!80} \vspace{0.2cm}  \textbf{Must} \newline & \vspace{0.2cm}  \textbf{Should} & \vspace{0.2cm}  \textbf{Could} &  \vspace{0.2cm} \textbf{Won't} \\ 
\hline
 Visning af afstanden på endt gåtur.   & Løbende tracking af rute på kort & Mulighed for at optage videorefleksion over gåtur & Notificering hvis ens pårørende har gået en tur \vspace{1cm} \\
\hline 
Mulighed for at dele rute med pårørende & Mulighed for at dele distance, rute og doneringsbeløb igennem Facebook &  Historik over ture og tilhørende doneringsbeløb & Mulighed for at dele refleksionsvideo  \vspace{1cm} \\
\hline
 & Som en bruger, vil jeg have at et beløb doneres til velgørenhed baseret på min gåtur &  &  \\
\hline
\end{tabular}
\caption{MoSCoW-analyse}
\label{tab:MosCoW}
\end{table}
\FloatBarrier


\section{User stories}
Ud fra den opstillede MoSCoW analyse opstilles user stories, som underbygger den ønskede agile udviklingsprocess.
De opbygges ud fra følgende skabelon: \\

\centerline{Som en <bruger> vil jeg <gøre noget> så <jeg kan opnå>}  \hfill

I Projektet er det valgt at fokusere på følgende 9 user stories:
\begin{enumerate}

\item Som en bruger, vil jeg kunne starte en gåtur.
\item Som en bruger, vil jeg kunne følge antallet af kilometer jeg har gået. 
\item Som en bruger vil jeg kunne stoppe gåturen, når jeg er færdig. 
\item Som en bruger, vil jeg kunne følge min gåtur på et kort. 
\item Som en bruger, vil jeg have at et pengebeløb doneres baseret på min gåtur. 
\item Som en bruger, vil jeg efter endt gåtur kunne se det donerede beløb baseret på min gåtur. 
\item Som en bruger, vil jeg efter endt gåtur kunne se det samlede antal kilometer, som jeg er gået. 
\item Som en bruger, vil jeg kunne logge på med min Facebook, så at det bliver nemt for at oprette en bruger. 
\item Som bruger, vil jeg efter endt gåtur kunne dele min gåtur på Facebook, så jeg kan inspirere andre til at gøre det samme. 
\item Som en bruger, vil jeg kunne se historik over alle mine tidligere gåture
\end{enumerate}

Følgende user stories ønskes udviklet igennem en iterativ process baseret på prioritering i MoSCoW anaylsen. Til udviklingen benyttes continous integration, som giver muligheden for at opdele projektet og arbejde mere effektivt på seperate features.